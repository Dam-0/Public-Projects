%!TEX program = xelatex

\documentclass[12pt, letterpaper]{article}
\usepackage[utf8]{inputenc}
\usepackage{graphicx, hyperref}


\usepackage{fontspec}
%\setsansfont[
%  BoldFont=AmaticSC-Bold.ttf
%]{AmaticSC-Regular.ttf}

%\Large{\sffamily Hello World}




%hyperref is a great package. Allows you to add in URL's and links to other places on the page

%tocbibind to enhance the table of contents

%fancyref is great for referencing labels for figures, tables and sections
%"in \fref{fig:image}." to get nice text like "in figure 11 on the next page."

\setlength{\parindent}{0em}
\setlength{\parskip}{1em}

%\graphicspath{ {$HOME/pictures/} }

\title{Example Document}
\author{Damien Peters}
\date{July 2021}

\begin{document}

\maketitle

This is an example document using \LaTeX{} for creating a document.


%This a comment

%This is an example of using font tools
This is an example of using \textbf{bold text} in a sentence.\\
This is an example of using \underline{underlined text} in a sentence.\\
This is an example of using \textit{italics text} in a sentence.\\
This is an example of using \textbf{\underline{underlined bold text}} in a sentence.
\par
%This is an example of using images


This is an example of using an image in \LaTeX{}.

\begin{figure}[h]
	\centering
	\href{wget https://hatrabbits.com/wp-content/uploads/2017/01/random.jpg}{random.jpg}
	\caption{A random picture}
	\label{fig:random1}
\end{figure}
\par
\newpage

This is example of using lists in \LaTeX{}.

\begin{itemize}
	\item This is the first dot point
	\item This can be of any length
	\begin{itemize}
		\item Nested dot
	\end{itemize}
\end{itemize}

\begin{enumerate}
	\item This is the first enumerated point
	\item This can be of any length
	\begin{enumerate}
		\item Nested enumerate
	\end{enumerate}
\end{enumerate}
\par

Table \ref{table:data} is an example of referenced \LaTeX{} elements.

\begin{table}[h!]
\centering
\begin{tabular}{||c c c||} 
 \hline
 ID & Name & Age \\ [0.5ex] 
 \hline\hline
 1 & Damo & !30 \\ 
 2 & Not Damo & old  \\
 3 & \textbf{Chad} & 21 \\ [1ex] 
 \hline
\end{tabular}
\caption{Table to test captions and labels}
\label{table:data}
\end{table}

\end{document}


